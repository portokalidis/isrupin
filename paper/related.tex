\section{Related}
\label{sec:related}

We can split CFI systems found in the literature into two main CFI approaches: 
Systems that rely on just binaries and do not require recompiling 
and those that require to be present in compilation time and act as 
another level in the build toolchain. 
Systems in the first category typically begin by disassembling the 
executable and then attempting to identify transfer control edges 
by constructing a CFG; some of those systems attempt to locate only 
forward edges while others look into return addresses as well. Systems 
from the second category since they have access during the build time, 
can generally produce more complete CFG since they have access to 
the source code but on the flip side require rebuilding of all modules 
in the software and that is not possible in closed source modules or shared 
libraries where source code is not freely available. Specifically for the shared 
libraries, this is another key point where work found in the literature forks 
since some systems support having shared libraries while others do not.



In the table below ``Alignment'' refers to protection in jumping in the middle 
of instructions due to x86 CISC dense architecture. ``Shared libraries'' represents 
the ability of the system to work with modules not necessarily protected by 
the technique and/or loaded at a later time. 
Specifically for Safedispatch~\cite{jang2014safedispatch} while it is not a  general CFI system
and targets vtables and hence C++ code; it was mentioned 
extensively in the related work of most CFI systems; while it captures only JOP style 
attacks targeting vtables it claims that this is one of the most important attack vectors, 
this claim is supported by Tice et al.~\cite{tice2014enforcing} were they report that 
vtable jumps correspond to 95-99\% of all Indirect Control Transfers. 

Everyone assume additional protections such as stack canaries and DEP or WOX 
and some form of ASLR.



\begin{table*}[tp]
\centering
\caption{Comparison Table}
\label{table:overview}
\begin{tabular}{lcc|c|c|c|c|c|c|c}
\cline{4-9}
        & \multicolumn{1}{l}{}                   & \multicolumn{1}{l|}{}          & \multicolumn{6}{c|}{Protects}               & \multicolumn{1}{l}{}             \\ \hline
\multicolumn{1}{|l|}{Name}           & \multicolumn{1}{c|}{Relocation Tables} & \multicolumn{1}{l|}{Compiling} & \multicolumn{1}{l|}{Vtable} & \multicolumn{1}{l|}{Indirect} & \multicolumn{1}{l|}{Direct} & \multicolumn{1}{l|}{Ret} & \multicolumn{1}{l|}{SharedLibraries} & \multicolumn{1}{l|}{Alignment} & \multicolumn{1}{l|}{Performance} \\ \hline
\multicolumn{1}{|l|}{CCFIR\cite{zhang2013practical}}          & \multicolumn{1}{c|}{X}                 & X & -                           & \checkmark & DEP                         &\checkmark& \checkmark       & \checkmark                            & \multicolumn{1}{c|}{3.6\%/8.1\%} \\ \hline
\multicolumn{1}{|l|}{binCFI\cite{zhang2013control}}         & \multicolumn{1}{c|}{two versions}      & X & \checkmark & \checkmark & DEP & \checkmark & \checkmark       & \checkmark & \multicolumn{1}{c|}{12\%-45\%}   \\ \hline
\multicolumn{1}{|l|}{ForwardEdgeCFI\cite{tice2014enforcing}} & \multicolumn{1}{c|}{-} & \checkmark & \checkmark & \checkmark & DEP & X & \checkmark       & \checkmark & \multicolumn{1}{c|}{2\%/8\%}     \\ \hline

\multicolumn{1}{|l|}{CFI\cite{Abadi:2005:CI:1102120.1102165}}            & \multicolumn{1}{c|}{\checkmark}                 & X & \checkmark & \checkmark & DEP & \checkmark & \checkmark       & \checkmark & \multicolumn{1}{c|}{15\%/46\%}   \\ \hline
\multicolumn{1}{|l|}{Safedispatch}   & \multicolumn{1}{c|}{-}                 & \checkmark & \checkmark                           & - & DEP                         & - & X       & \checkmark & \multicolumn{1}{c|}{2.1\%}       \\ \hline
\end{tabular}
\end{table*}